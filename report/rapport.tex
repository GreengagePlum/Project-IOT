\RequirePackage[l2tabu, orthodox]{nag}
\documentclass[12pt]{article}
\usepackage[T1]{fontenc}
\usepackage[utf8]{inputenc}
\usepackage[french]{babel}
\usepackage{amsthm,amssymb,amsmath,xcolor}
\usepackage{setspace}
% \doublespacing
\usepackage{geometry}
\geometry{
    a4paper,
    total={170mm,257mm},
}
\usepackage{graphicx}
\graphicspath{ {./} }
\usepackage{microtype}
\usepackage{todonotes}
\usepackage{hyperref}
\hypersetup{
    colorlinks=true,
    linkcolor=blue,
    filecolor=magenta,
    urlcolor=cyan,
}

\author{Efe ERKEN, Sylvain PRANDO}
\date{\today}
\title{Rapport Projet ``Réseau de capteurs''}

\begin{document}
\maketitle

\begin{abstract}
    Dans le cadre de l'UE Interface des Objets Communiquants (IOC) du parcours de master informatique SESI 2024-25 à la
    Sorbonne, nous avons été donné à faire ce projet de programmation pour réaliser un réseau de capteurs "IOT" avec une
    interface web. Plus concrétement, notre but était de créer un système où un ou plusieurs cartes \texttt{ESP32} équippées avec des
    capteurs comme une photorésistance ainsi que d'autres composants comme une LED et un bouton poussoir communiquent avec
    un mini serveur logé sur une carte Raspberry Pi 3 pour d'un côté transmettre vers ce serveur les données lues depuis les
    capteurs à intervalles réguliers afin de les stocker, puis afficher sur une interface web, et d'un autre côté contrôler les cartes \texttt{ESP32} (par
    exemple leur LED) dans l'autre sens depuis l'interface web. Tout cela en utilisant le Wi-Fi sur ces cartes ainsi que le
    protocole \texttt{MQTT} pour faire communiquer tous les appareils.
\end{abstract}

\section{Objectifs de notre projet}
Pour répondre aux demandes décrites dans le projet, nous avons donc fixé les objectifs suivants :
\begin{itemize}
    \item réalisation d'une interface web en tant que site web
    \item réalisation d'un serveur web pour héberger le site web
    \item réalisation d'un client \texttt{MQTT} côté serveur pour communiquer avec les \texttt{ESP32}
    \item configuration d'une base de données sur le serveur
    \item réalisation du code des appareils \texttt{ESP32}
\end{itemize} \hfill \break

Notre but premier était donc d'abord deux choses. Premièrement, pouvoir stocker les valeur lues depuis les capteurs sur le serveur pour les afficher sur l'interface web en tant que historique.
Et deuxièmement, pouvoir contrôler au moins la LED dans le sens inverse depuis l'interface web.

Cela allait nécessiter au minimum une page pour visualiser les données stockées dans la base de données et un contrôle
(bouton, case à cocher) pour manipuler l'état de la LED de chaque appareil.

Mais avec l'intention de faire plus, le potentiel temps réel du projet nous a attiré l'attention. En plus des demandes du sujet,
nous avons voulu aussi réaliser une page en plus dans notre site web dédié au temps réel. En effet, notre but était de
pouvoir afficher les changements d'état des cartes \texttt{ESP32} (tels que le bouton poussoir et la photorésistance)
instantanément sur l'interface sans même de rafraichir la page sur son navigateur web.

\section{Choix de conception}

bla

\section{Choix d'implémentation}

bla

\section{Difficultés}

bla

\section{Fonctionnalités complétées}

bla

\section{Fonctionnalités incomplètes}

bla

\section{Lacunes et améliorations possibles}

bla

\section{Usage}

bla

\begin{center}
	\includegraphics[scale=0.2]{placeholder.png}
\end{center}
\href{https://dbeaver.io/}{\texttt{DBeaver}}

\end{document}

